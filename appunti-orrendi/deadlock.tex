% Created 2022-07-07 Thu 12:22
% Intended LaTeX compiler: pdflatex
\documentclass[11pt]{article}
\usepackage[utf8]{inputenc}
\usepackage[T1]{fontenc}
\usepackage{graphicx}
\usepackage{longtable}
\usepackage{wrapfig}
\usepackage{rotating}
\usepackage[normalem]{ulem}
\usepackage{amsmath}
\usepackage{amssymb}
\usepackage{capt-of}
\usepackage{hyperref}
\date{\today}
\title{}
\hypersetup{
 pdfauthor={},
 pdftitle={},
 pdfkeywords={},
 pdfsubject={},
 pdfcreator={Emacs 28.1 (Org mode 9.5.2)}, 
 pdflang={English}}
\begin{document}

\tableofcontents

\section{Deadlocks}
\label{sec:org8344d14}

Facciamo un esempio stupido
Devo prendere due libri dalla biblioteca per un esame, a Luigi servono gli stessi due
libri, io prendo il libro A, Luigi prende il libro B, ora io dovrei prendere il libro
B, ma l'ha già preso Luigi, ok, aspetto che lo renda, intanto mi tengo questo.
Ora a Luigi serve il libro A, ma l'ho già preso io, ok, aspetta che io lo renda,
intanto si tiene quello.
Bocciamo entrambi.

In un computer spesso si vuole dare accesso esclusivo a una risorsa, magari è
un file o un handle per non so cosa, e non voglio race condition quindi ci accede
solo un processo alla volta, magari è il permesso di usare la stampante e non vorrei
avere due processi iniseme che mandano roba alla stampante, che è già stupida di
suo. Queste situazioni però possono dare il via all'effetto Luigi illustrato di
sopra, come si capisce dal nome della sezione il termine tecnico per l'effetto Luigi
è \textbf{DEADLOCK}, che suona anche più figo

Il deadlock tra due processi succede quando io sto aspettando il libro di Luigi, ma
Luigi sta aspettando il mio libro, visto che ambo i tizii in deadlock stanno
aspettando caio, che è in deadlock, non se ne esce.

Come se non bastasse i deadlock possono accadere anche con più di due processi, se
c'ho una combriccola di tizii, caii, e sempronii, che stanno tutti aspettando un
altro tizio, caio, o sempronio in deadlock, non ne esco ugualmente.

In pseudo-matematichese, un gruppo di processi è in deadlock se ogniuno di quei
processi sta aspettando un processo in deadlock

Dato lo pseudo-matematichese non stupirà che abbiamo matematichese
dato il matematichese non stupirà che qualcuno si è messo a fare teoremi su sta roba,
ahime.

Queste sono delle condizioni necessarie al potersi ritrovare con un deadlock

\begin{itemize}
\item una risorsa o ce l'ha un solo tizio o è pigliabile, che ce l'abbia io o che ce
l'abbia un qualche processo sono quindi conidizioni che si
escludono a vicenda, c'è una \textbf{mutua esclusione}, facile sennò la
dai due tizii che la vogliono e non sto più aspettando st'altro
stronzo, la definizione casca.

\item posso tenermi le risorse che ho acquisito mentre aspetto che mi arrivino le altre
risorse che mi servono, se usassi e rilasciassi una risorsa alla volta
quella che ho prima o poi la rilascio indipendentemente da che
fanno gli altri, che non sto aspettando niente, poi chiunque mi
stia aspettando ce l'avrà poi, non sembra propio un deadlock.

\item niente prelazione di risorse, non possono venire a fottermi il
libro per darlo a Luigi, altrimenti il sistema potrebbe
sbarazzarsi molto tranquillamente dei deadlock.

\item c'è un girotondo chiuso di gente che sta aspettando quello davanti
nel girotondo, se non ci fossero girotondi ci sarebbe quello in
cima alla coda che non sta aspettando nessuno e si leva, poi quello
che era dietro ora è in cima e si leva pure lui, poi si levano
tutti, non ha molto l'aria da deadlock.
\end{itemize}

prendiamo l'ultima condizione, in pseudo-matematichese possiamo dire
che
\begin{quote}
c'è un qualche ciclo nel grafo con i processi come nodi e con \(u \to
v\) se \(u\) gradirebbe che \(v\) si desse una mossa
\end{quote}
\end{document}